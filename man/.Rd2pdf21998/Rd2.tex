\documentclass[letterpaper]{book}
\usepackage[times,inconsolata,hyper]{Rd}
\usepackage{makeidx}
\usepackage[latin1]{inputenc} % @SET ENCODING@
% \usepackage{graphicx} % @USE GRAPHICX@
\makeindex{}
\begin{document}
\chapter*{}
\begin{center}
{\textbf{\huge Adjacency list from progrex index}}
\par\bigskip{\large \today}
\end{center}
\inputencoding{utf8}
\HeaderA{adjl\_from\_adjmat}{From adjacency matrix to adjacency list}{adjl.Rul.from.Rul.adjmat}
%
\begin{Description}\relax
This function is able to transform a matrix of distances into an adjacency list that can be used for the analysis pipeline.
Please remember that \code{gen\_progindex} accepts only minimum spanning trees.
\end{Description}
%
\begin{Usage}
\begin{verbatim}
adjl_from_adjmat(adj_m)
\end{verbatim}
\end{Usage}
%
\begin{Arguments}
\begin{ldescription}
\item[\code{adj\_m}] Input matrix (adjacency matrix).
\end{ldescription}
\end{Arguments}
%
\begin{Value}
A list of three elements: degree list, connectivity matrix and weights.
\end{Value}
%
\begin{SeeAlso}\relax
\code{\LinkA{gen\_progindex}{gen.Rul.progindex}}, \code{\LinkA{mst\_from\_trj}{mst.Rul.from.Rul.trj}}.
\end{SeeAlso}
